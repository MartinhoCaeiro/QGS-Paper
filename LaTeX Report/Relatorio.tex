\documentclass[a4paper]{article}
% Pacotes necessários
\usepackage[portuguese]{babel}
\usepackage[backend=biber, style=apa, citestyle=apa, language=portuguese]{biblatex}
\usepackage{csquotes}
\addbibresource{Recursos/referencias.bib}

\usepackage{amsmath}
\usepackage{graphicx}
\usepackage{subcaption}
\usepackage{setspace}
\usepackage{siunitx} % Required for alignment
\sisetup{
  round-mode          = places, % Rounds numbers
  round-precision     = 2, % to 2 places
}
\usepackage{enumerate}
\usepackage{enumitem}
\usepackage{amsmath}
\usepackage{karnaugh-map}
\usepackage[section]{placeins}
\usepackage{geometry}
\usepackage{amssymb}
\usepackage{titling}
\usepackage[T1]{fontenc}
\usepackage{float}
\usepackage[hidelinks]{hyperref}
\usepackage{xcolor}
\usepackage{indentfirst}
\usepackage{array}
\usepackage{soul}
\usepackage{afterpage}
\newcolumntype{P}[1]{>{\centering\arraybackslash}p{#1}}
\onehalfspacing

% Comando para criar uma página vazia
\newcommand\myemptypage{
    \null
    \thispagestyle{empty}
    \addtocounter{page}{-1}
    \newpage
}

% Página de título principal
\newcommand{\firsttitlepage}{
    \begin{titlepage}
        \centering
        \vspace*{1cm}
        
        % Logos superior
        \begin{figure}[h!]
            \centering
            \includegraphics[width=5cm]{Recursos/LOGO_IPB} % Substitua pelo caminho da imagem
            \vspace{0.5cm}
        \end{figure}

        % Informações da instituição
        \large\textbf{INSTITUTO POLITÉCNICO DE BEJA} \\
        \large\textbf{Escola Superior de Tecnologia e Gestão} \\
        \large\textbf{Mestrado em Engenharia de Segurança Informática} \\
        \large\textbf{Direito na Segurança Informática e no Cibercrime} \\
        
        \vspace{2cm}
        
        % Título do projeto
        {\LARGE \textbf{Unveiling Samsung Quantum Galaxy: Securing Smartphones With Quantum and Post-Quantum Cryptography}} \\
        
        \vspace{1.0cm}
        
        % Autores
        \large Martinho José Novo Caeiro - 23917 \\
        \large Paulo António Tavares Abade - 23919 \\
        
        \vfill
        
        % Logo inferior
        \begin{figure}[h!]
            \centering
            \includegraphics[width=5cm]{Recursos/IPBejaESTIG.jpg} % Substitua pelo caminho da imagem
        \end{figure}
        
        \vspace{1cm}
        
        % Local e data
        {\large Beja, janeiro de 2026}
    \end{titlepage}
}

\newcommand{\secondtitlepage}{
    \begin{titlepage}
        \centering
        \vspace*{1cm}
        
        % Informações da instituição
        \large\textbf{INSTITUTO POLITÉCNICO DE BEJA} \\
        \large\textbf{Escola Superior de Tecnologia e Gestão} \\
        \large\textbf{Mestrado em Engenharia de Segurança Informática} \\
        \large\textbf{Direito na Segurança Informática e no Cibercrime} \\
        
        \vspace{2cm}
        
        % Título do projeto
        {\LARGE \textbf{Unveiling Samsung Quantum Galaxy: Securing Smartphones With Quantum and Post-Quantum Cryptography}} \\
        
        \vspace{1.5cm}
        
        % Autores
        \large Martinho José Novo Caeiro - 23917 \\
        \large Paulo António Tavares Abade - 23919 \\

        \vspace{2cm}

        % Orientador
        \large Orientador:  Rui Miguel Silva \& Daniel José Da Graça Peceguinha Franco \\
        
        \vfill
        
        % Local e data
        {\large Beja, janeiro de 2026}
    \end{titlepage}
}

\begin{document}


\pagenumbering{gobble} % Oculta numeração da página

% Primeira página de título
\firsttitlepage

\secondtitlepage

\vspace{1cm}
\newpage
%--------------------------------------------------------------------------------------------------------------------------------------

\vspace{1cm}

% Início do conteúdo do relatório
\newpage
\doublespacing
\doublespacing

\newpage
\pagenumbering{arabic}

\section{Article Analysis}\label{intro}
The analysis of the article will be make through the answers to the questions on the \textit{template} that was provided by the professor.

\subsection{What is the name of the research?}
The name of the research is "Unveiling Samsung Quantum Galaxy: Securing Smartphones With Quantum and Post-Quantum Cryptography".
\subsection{Where the research was published?}
The research was published in the journal "IEEE Access", which is a multidisciplinary, open access journal of the Institute of Electrical and Electronics Engineers (IEEE).
\subsection{Who is the author?}
The name of the author is Omar Alibrahim, a researcher with support from Kuwait Foundation for the Advancemente of Sciences (KFAS). The 
research was published on 2nd of May 2025.
\subsection{What's the theory of research?}

\subsection{What's the problem being addressed in this study?}

\subsection{What are the objectives addressed?}

\subsection{What are the study's strength points?}

\subsection{What are the study's weaknesses?}

\subsection{Which type of methodology was adopted?}

\subsection{Clearly identify the adopted research design method.}

\subsection{Briefly explain the variable of analysis used on this study.}

\subsection{Cleary identify the criteria/equations that were used to validade results.}

\subsection{Critically analyze the presented results}

\subsection{What are the conclusions presented by the author?}

\subsection{What is the contribution to the existing knowledge?}

\subsection{Can you accept the findings as true? Discuss any failing and shortcomings of the method used to support the findings.}

\subsection{Summarize your conclusion on the analyzing the benchmark.}

\subsection{What is the main gap that you have identified on this study, and that you are willing to address on your research?}

\subsection{Which part of the benchmark will you use in your study, in order to compare your results with their outcomes?}

\subsection{Conclude by summarizing the Research Problem + Research Gap+ Purpose of the study (MAXIMIUM 40 WORDS)}
%---------------------------------------------------------------------------------------------------------------------------

\newpage
\renewcommand{\refname}{Bibliografia} % Para artigos
\renewcommand{\bibname}{Bibliografia} % Para livros e relatórios
\addcontentsline{toc}{section}{Bibliografia} % Adiciona a Bibliografia ao índice
\printbibliography

\end{document}
